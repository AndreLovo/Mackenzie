% Options for packages loaded elsewhere
\PassOptionsToPackage{unicode}{hyperref}
\PassOptionsToPackage{hyphens}{url}
%
\documentclass[
]{article}
\usepackage{lmodern}
\usepackage{amssymb,amsmath}
\usepackage{ifxetex,ifluatex}
\ifnum 0\ifxetex 1\fi\ifluatex 1\fi=0 % if pdftex
  \usepackage[T1]{fontenc}
  \usepackage[utf8]{inputenc}
  \usepackage{textcomp} % provide euro and other symbols
\else % if luatex or xetex
  \usepackage{unicode-math}
  \defaultfontfeatures{Scale=MatchLowercase}
  \defaultfontfeatures[\rmfamily]{Ligatures=TeX,Scale=1}
\fi
% Use upquote if available, for straight quotes in verbatim environments
\IfFileExists{upquote.sty}{\usepackage{upquote}}{}
\IfFileExists{microtype.sty}{% use microtype if available
  \usepackage[]{microtype}
  \UseMicrotypeSet[protrusion]{basicmath} % disable protrusion for tt fonts
}{}
\makeatletter
\@ifundefined{KOMAClassName}{% if non-KOMA class
  \IfFileExists{parskip.sty}{%
    \usepackage{parskip}
  }{% else
    \setlength{\parindent}{0pt}
    \setlength{\parskip}{6pt plus 2pt minus 1pt}}
}{% if KOMA class
  \KOMAoptions{parskip=half}}
\makeatother
\usepackage{xcolor}
\IfFileExists{xurl.sty}{\usepackage{xurl}}{} % add URL line breaks if available
\IfFileExists{bookmark.sty}{\usepackage{bookmark}}{\usepackage{hyperref}}
\hypersetup{
  pdftitle={Trilha3-Explorando a Base de Dados diamonds do pacote ggplot2},
  pdfauthor={André Gustavo Silva Lovo},
  hidelinks,
  pdfcreator={LaTeX via pandoc}}
\urlstyle{same} % disable monospaced font for URLs
\usepackage[margin=1in]{geometry}
\usepackage{color}
\usepackage{fancyvrb}
\newcommand{\VerbBar}{|}
\newcommand{\VERB}{\Verb[commandchars=\\\{\}]}
\DefineVerbatimEnvironment{Highlighting}{Verbatim}{commandchars=\\\{\}}
% Add ',fontsize=\small' for more characters per line
\usepackage{framed}
\definecolor{shadecolor}{RGB}{248,248,248}
\newenvironment{Shaded}{\begin{snugshade}}{\end{snugshade}}
\newcommand{\AlertTok}[1]{\textcolor[rgb]{0.94,0.16,0.16}{#1}}
\newcommand{\AnnotationTok}[1]{\textcolor[rgb]{0.56,0.35,0.01}{\textbf{\textit{#1}}}}
\newcommand{\AttributeTok}[1]{\textcolor[rgb]{0.77,0.63,0.00}{#1}}
\newcommand{\BaseNTok}[1]{\textcolor[rgb]{0.00,0.00,0.81}{#1}}
\newcommand{\BuiltInTok}[1]{#1}
\newcommand{\CharTok}[1]{\textcolor[rgb]{0.31,0.60,0.02}{#1}}
\newcommand{\CommentTok}[1]{\textcolor[rgb]{0.56,0.35,0.01}{\textit{#1}}}
\newcommand{\CommentVarTok}[1]{\textcolor[rgb]{0.56,0.35,0.01}{\textbf{\textit{#1}}}}
\newcommand{\ConstantTok}[1]{\textcolor[rgb]{0.00,0.00,0.00}{#1}}
\newcommand{\ControlFlowTok}[1]{\textcolor[rgb]{0.13,0.29,0.53}{\textbf{#1}}}
\newcommand{\DataTypeTok}[1]{\textcolor[rgb]{0.13,0.29,0.53}{#1}}
\newcommand{\DecValTok}[1]{\textcolor[rgb]{0.00,0.00,0.81}{#1}}
\newcommand{\DocumentationTok}[1]{\textcolor[rgb]{0.56,0.35,0.01}{\textbf{\textit{#1}}}}
\newcommand{\ErrorTok}[1]{\textcolor[rgb]{0.64,0.00,0.00}{\textbf{#1}}}
\newcommand{\ExtensionTok}[1]{#1}
\newcommand{\FloatTok}[1]{\textcolor[rgb]{0.00,0.00,0.81}{#1}}
\newcommand{\FunctionTok}[1]{\textcolor[rgb]{0.00,0.00,0.00}{#1}}
\newcommand{\ImportTok}[1]{#1}
\newcommand{\InformationTok}[1]{\textcolor[rgb]{0.56,0.35,0.01}{\textbf{\textit{#1}}}}
\newcommand{\KeywordTok}[1]{\textcolor[rgb]{0.13,0.29,0.53}{\textbf{#1}}}
\newcommand{\NormalTok}[1]{#1}
\newcommand{\OperatorTok}[1]{\textcolor[rgb]{0.81,0.36,0.00}{\textbf{#1}}}
\newcommand{\OtherTok}[1]{\textcolor[rgb]{0.56,0.35,0.01}{#1}}
\newcommand{\PreprocessorTok}[1]{\textcolor[rgb]{0.56,0.35,0.01}{\textit{#1}}}
\newcommand{\RegionMarkerTok}[1]{#1}
\newcommand{\SpecialCharTok}[1]{\textcolor[rgb]{0.00,0.00,0.00}{#1}}
\newcommand{\SpecialStringTok}[1]{\textcolor[rgb]{0.31,0.60,0.02}{#1}}
\newcommand{\StringTok}[1]{\textcolor[rgb]{0.31,0.60,0.02}{#1}}
\newcommand{\VariableTok}[1]{\textcolor[rgb]{0.00,0.00,0.00}{#1}}
\newcommand{\VerbatimStringTok}[1]{\textcolor[rgb]{0.31,0.60,0.02}{#1}}
\newcommand{\WarningTok}[1]{\textcolor[rgb]{0.56,0.35,0.01}{\textbf{\textit{#1}}}}
\usepackage{longtable,booktabs}
% Correct order of tables after \paragraph or \subparagraph
\usepackage{etoolbox}
\makeatletter
\patchcmd\longtable{\par}{\if@noskipsec\mbox{}\fi\par}{}{}
\makeatother
% Allow footnotes in longtable head/foot
\IfFileExists{footnotehyper.sty}{\usepackage{footnotehyper}}{\usepackage{footnote}}
\makesavenoteenv{longtable}
\usepackage{graphicx,grffile}
\makeatletter
\def\maxwidth{\ifdim\Gin@nat@width>\linewidth\linewidth\else\Gin@nat@width\fi}
\def\maxheight{\ifdim\Gin@nat@height>\textheight\textheight\else\Gin@nat@height\fi}
\makeatother
% Scale images if necessary, so that they will not overflow the page
% margins by default, and it is still possible to overwrite the defaults
% using explicit options in \includegraphics[width, height, ...]{}
\setkeys{Gin}{width=\maxwidth,height=\maxheight,keepaspectratio}
% Set default figure placement to htbp
\makeatletter
\def\fps@figure{htbp}
\makeatother
\setlength{\emergencystretch}{3em} % prevent overfull lines
\providecommand{\tightlist}{%
  \setlength{\itemsep}{0pt}\setlength{\parskip}{0pt}}
\setcounter{secnumdepth}{-\maxdimen} % remove section numbering

\title{Trilha3-Explorando a Base de Dados diamonds do pacote ggplot2}
\author{André Gustavo Silva Lovo}
\date{07/11/2020}

\begin{document}
\maketitle

\hypertarget{trilha3}{%
\section{Trilha3}\label{trilha3}}

\begin{Shaded}
\begin{Highlighting}[]
\CommentTok{#install.packages("readr", dependencies = TRUE)}
\KeywordTok{library}\NormalTok{(readr) }
\CommentTok{#install.packages("data.table", dependencies = TRUE)}
\KeywordTok{library}\NormalTok{(data.table)}
\CommentTok{#install.packages("knitr")}
\KeywordTok{library}\NormalTok{(knitr)}
\CommentTok{#install.packages("tinytex")}
\NormalTok{tinytex}\OperatorTok{::}\KeywordTok{install_tinytex}\NormalTok{()}
\CommentTok{#install.packages("ggplot2", dependencies = TRUE)}
\CommentTok{#install.packages("dplyr", dependencies = TRUE)}
\KeywordTok{library}\NormalTok{(dplyr) }
\end{Highlighting}
\end{Shaded}

\begin{verbatim}
## 
## Attaching package: 'dplyr'
\end{verbatim}

\begin{verbatim}
## The following objects are masked from 'package:data.table':
## 
##     between, first, last
\end{verbatim}

\begin{verbatim}
## The following objects are masked from 'package:stats':
## 
##     filter, lag
\end{verbatim}

\begin{verbatim}
## The following objects are masked from 'package:base':
## 
##     intersect, setdiff, setequal, union
\end{verbatim}

\begin{Shaded}
\begin{Highlighting}[]
\CommentTok{#install.packages("tidyr", dependencies = TRUE)}
\KeywordTok{library}\NormalTok{(tidyr)}
\end{Highlighting}
\end{Shaded}

\hypertarget{para-esta-atividade-de-aed-anuxe1lise-exploratuxf3ria-de-dados-vamos-seguir-o-seguinte-checklist}{%
\subsection{Para esta atividade de AED (Análise Exploratória de Dados)
vamos seguir o seguinte
checklist:}\label{para-esta-atividade-de-aed-anuxe1lise-exploratuxf3ria-de-dados-vamos-seguir-o-seguinte-checklist}}

\begin{figure}
\centering
\includegraphics{check_list.jpg}
\caption{checklist}
\end{figure}

\begin{Shaded}
\begin{Highlighting}[]
\KeywordTok{library}\NormalTok{(ggplot2)}
\NormalTok{diamantes <-}\StringTok{ }\KeywordTok{data}\NormalTok{(diamonds)}
\end{Highlighting}
\end{Shaded}

\hypertarget{dados-e-informauxe7uxf5es-do-banco-de-dados-diamonds}{%
\section{Dados e informações do Banco de dados
diamonds:}\label{dados-e-informauxe7uxf5es-do-banco-de-dados-diamonds}}

price: preço em dólares americanos (\$326--\$18,823)\\
carat: peso do diamente (0.2--5.01)\\
cut:qualidade do corte (Fair, Good, Very Good, Premium, Ideal)\\
color: cor do diamante, indo de J (pior) a D (melhor)\\
clarity: medida de quão claro é o diamante (I1 (pior), SI2, SI1, VS2,
VS1, VVS2, VVS1, IF (melhor))\\
x: comprimento em mm (0--10.74)\\
y: largura em mm (0--58.9)\\
z: profundidade em mm (0--31.8)\\
depth: percentual de profundidade total = z / mean(x, y) = 2 * z / (x +
y) (43--79)\\
table: largura do topo do diamante relativo ao ponto mais largo (43--95)

Como pode ser observado a tabela apresenta as seguintes variáveis para a
quantificação e qualificação de uma tabela com características para os
diamantes:\\
Preço - peso - qualidade do corte - claridade - comprimento - largura -
profundidade - profundidade total em porcentagem - relação de largura
entre o topo e a parte mais larga.

\begin{Shaded}
\begin{Highlighting}[]
\KeywordTok{dim}\NormalTok{(diamonds)}
\end{Highlighting}
\end{Shaded}

\begin{verbatim}
## [1] 53940    10
\end{verbatim}

\hypertarget{investigando-e-analisando-a-base-de-dados}{%
\subsection{Investigando e analisando a base de
dados:}\label{investigando-e-analisando-a-base-de-dados}}

\hypertarget{qual-uxe9-a-estrutura-do-conjunto-de-dados-diamantes}{%
\subsubsection{Qual é a estrutura do conjunto de dados
``diamantes''?}\label{qual-uxe9-a-estrutura-do-conjunto-de-dados-diamantes}}

\begin{Shaded}
\begin{Highlighting}[]
\KeywordTok{View}\NormalTok{ (diamonds)}
\end{Highlighting}
\end{Shaded}

\begin{Shaded}
\begin{Highlighting}[]
\KeywordTok{typeof}\NormalTok{(diamonds}\OperatorTok{$}\NormalTok{price)}
\end{Highlighting}
\end{Shaded}

\begin{verbatim}
## [1] "integer"
\end{verbatim}

\begin{Shaded}
\begin{Highlighting}[]
\KeywordTok{typeof}\NormalTok{(diamonds}\OperatorTok{$}\NormalTok{carat)}
\end{Highlighting}
\end{Shaded}

\begin{verbatim}
## [1] "double"
\end{verbatim}

\begin{Shaded}
\begin{Highlighting}[]
\KeywordTok{typeof}\NormalTok{(diamonds}\OperatorTok{$}\NormalTok{cut)}
\end{Highlighting}
\end{Shaded}

\begin{verbatim}
## [1] "integer"
\end{verbatim}

\begin{Shaded}
\begin{Highlighting}[]
\KeywordTok{typeof}\NormalTok{(diamonds}\OperatorTok{$}\NormalTok{color)}
\end{Highlighting}
\end{Shaded}

\begin{verbatim}
## [1] "integer"
\end{verbatim}

\begin{Shaded}
\begin{Highlighting}[]
\KeywordTok{typeof}\NormalTok{(diamonds}\OperatorTok{$}\NormalTok{clarity)}
\end{Highlighting}
\end{Shaded}

\begin{verbatim}
## [1] "integer"
\end{verbatim}

\begin{Shaded}
\begin{Highlighting}[]
\KeywordTok{typeof}\NormalTok{(diamonds}\OperatorTok{$}\NormalTok{depth)}
\end{Highlighting}
\end{Shaded}

\begin{verbatim}
## [1] "double"
\end{verbatim}

\begin{Shaded}
\begin{Highlighting}[]
\KeywordTok{typeof}\NormalTok{(diamonds}\OperatorTok{$}\NormalTok{table)}
\end{Highlighting}
\end{Shaded}

\begin{verbatim}
## [1] "double"
\end{verbatim}

\begin{Shaded}
\begin{Highlighting}[]
\KeywordTok{typeof}\NormalTok{(diamonds}\OperatorTok{$}\NormalTok{x)}
\end{Highlighting}
\end{Shaded}

\begin{verbatim}
## [1] "double"
\end{verbatim}

\begin{Shaded}
\begin{Highlighting}[]
\KeywordTok{typeof}\NormalTok{(diamonds}\OperatorTok{$}\NormalTok{y)}
\end{Highlighting}
\end{Shaded}

\begin{verbatim}
## [1] "double"
\end{verbatim}

\begin{Shaded}
\begin{Highlighting}[]
\KeywordTok{typeof}\NormalTok{(diamonds}\OperatorTok{$}\NormalTok{z)}
\end{Highlighting}
\end{Shaded}

\begin{verbatim}
## [1] "double"
\end{verbatim}

\begin{Shaded}
\begin{Highlighting}[]
\KeywordTok{str}\NormalTok{(diamonds)}
\end{Highlighting}
\end{Shaded}

\begin{verbatim}
## tibble [53,940 x 10] (S3: tbl_df/tbl/data.frame)
##  $ carat  : num [1:53940] 0.23 0.21 0.23 0.29 0.31 0.24 0.24 0.26 0.22 0.23 ...
##  $ cut    : Ord.factor w/ 5 levels "Fair"<"Good"<..: 5 4 2 4 2 3 3 3 1 3 ...
##  $ color  : Ord.factor w/ 7 levels "D"<"E"<"F"<"G"<..: 2 2 2 6 7 7 6 5 2 5 ...
##  $ clarity: Ord.factor w/ 8 levels "I1"<"SI2"<"SI1"<..: 2 3 5 4 2 6 7 3 4 5 ...
##  $ depth  : num [1:53940] 61.5 59.8 56.9 62.4 63.3 62.8 62.3 61.9 65.1 59.4 ...
##  $ table  : num [1:53940] 55 61 65 58 58 57 57 55 61 61 ...
##  $ price  : int [1:53940] 326 326 327 334 335 336 336 337 337 338 ...
##  $ x      : num [1:53940] 3.95 3.89 4.05 4.2 4.34 3.94 3.95 4.07 3.87 4 ...
##  $ y      : num [1:53940] 3.98 3.84 4.07 4.23 4.35 3.96 3.98 4.11 3.78 4.05 ...
##  $ z      : num [1:53940] 2.43 2.31 2.31 2.63 2.75 2.48 2.47 2.53 2.49 2.39 ...
\end{verbatim}

\begin{Shaded}
\begin{Highlighting}[]
\KeywordTok{View}\NormalTok{(diamonds)}
\end{Highlighting}
\end{Shaded}

\hypertarget{explore-a-parte-inicial-e-a-final-do-conjunto-de-dados.}{%
\subsubsection{Explore a parte inicial e a final do conjunto de
dados.}\label{explore-a-parte-inicial-e-a-final-do-conjunto-de-dados.}}

\begin{Shaded}
\begin{Highlighting}[]
\KeywordTok{head}\NormalTok{(diamonds,}\DecValTok{5}\NormalTok{)}
\end{Highlighting}
\end{Shaded}

\begin{verbatim}
## # A tibble: 5 x 10
##   carat cut     color clarity depth table price     x     y     z
##   <dbl> <ord>   <ord> <ord>   <dbl> <dbl> <int> <dbl> <dbl> <dbl>
## 1 0.23  Ideal   E     SI2      61.5    55   326  3.95  3.98  2.43
## 2 0.21  Premium E     SI1      59.8    61   326  3.89  3.84  2.31
## 3 0.23  Good    E     VS1      56.9    65   327  4.05  4.07  2.31
## 4 0.290 Premium I     VS2      62.4    58   334  4.2   4.23  2.63
## 5 0.31  Good    J     SI2      63.3    58   335  4.34  4.35  2.75
\end{verbatim}

\begin{Shaded}
\begin{Highlighting}[]
\KeywordTok{tail}\NormalTok{(diamonds,}\DecValTok{10}\NormalTok{)}
\end{Highlighting}
\end{Shaded}

\begin{verbatim}
## # A tibble: 10 x 10
##    carat cut       color clarity depth table price     x     y     z
##    <dbl> <ord>     <ord> <ord>   <dbl> <dbl> <int> <dbl> <dbl> <dbl>
##  1  0.71 Premium   E     SI1      60.5    55  2756  5.79  5.74  3.49
##  2  0.71 Premium   F     SI1      59.8    62  2756  5.74  5.73  3.43
##  3  0.7  Very Good E     VS2      60.5    59  2757  5.71  5.76  3.47
##  4  0.7  Very Good E     VS2      61.2    59  2757  5.69  5.72  3.49
##  5  0.72 Premium   D     SI1      62.7    59  2757  5.69  5.73  3.58
##  6  0.72 Ideal     D     SI1      60.8    57  2757  5.75  5.76  3.5 
##  7  0.72 Good      D     SI1      63.1    55  2757  5.69  5.75  3.61
##  8  0.7  Very Good D     SI1      62.8    60  2757  5.66  5.68  3.56
##  9  0.86 Premium   H     SI2      61      58  2757  6.15  6.12  3.74
## 10  0.75 Ideal     D     SI2      62.2    55  2757  5.83  5.87  3.64
\end{verbatim}

\hypertarget{fauxe7a-alguns-sumuxe1rios-estatuxedsticos-para-entender-melhor-a-base-de-dados.}{%
\subsubsection{Faça alguns sumários estatísticos para entender melhor a
base de
dados.}\label{fauxe7a-alguns-sumuxe1rios-estatuxedsticos-para-entender-melhor-a-base-de-dados.}}

\begin{Shaded}
\begin{Highlighting}[]
\KeywordTok{summary}\NormalTok{(diamonds)}
\end{Highlighting}
\end{Shaded}

\begin{verbatim}
##      carat               cut        color        clarity          depth      
##  Min.   :0.2000   Fair     : 1610   D: 6775   SI1    :13065   Min.   :43.00  
##  1st Qu.:0.4000   Good     : 4906   E: 9797   VS2    :12258   1st Qu.:61.00  
##  Median :0.7000   Very Good:12082   F: 9542   SI2    : 9194   Median :61.80  
##  Mean   :0.7979   Premium  :13791   G:11292   VS1    : 8171   Mean   :61.75  
##  3rd Qu.:1.0400   Ideal    :21551   H: 8304   VVS2   : 5066   3rd Qu.:62.50  
##  Max.   :5.0100                     I: 5422   VVS1   : 3655   Max.   :79.00  
##                                     J: 2808   (Other): 2531                  
##      table           price             x                y         
##  Min.   :43.00   Min.   :  326   Min.   : 0.000   Min.   : 0.000  
##  1st Qu.:56.00   1st Qu.:  950   1st Qu.: 4.710   1st Qu.: 4.720  
##  Median :57.00   Median : 2401   Median : 5.700   Median : 5.710  
##  Mean   :57.46   Mean   : 3933   Mean   : 5.731   Mean   : 5.735  
##  3rd Qu.:59.00   3rd Qu.: 5324   3rd Qu.: 6.540   3rd Qu.: 6.540  
##  Max.   :95.00   Max.   :18823   Max.   :10.740   Max.   :58.900  
##                                                                   
##        z         
##  Min.   : 0.000  
##  1st Qu.: 2.910  
##  Median : 3.530  
##  Mean   : 3.539  
##  3rd Qu.: 4.040  
##  Max.   :31.800  
## 
\end{verbatim}

\hypertarget{analisando-a-tabela-e-suas-colunas}{%
\subsection{Analisando a tabela e suas
colunas:}\label{analisando-a-tabela-e-suas-colunas}}

Coluna price: preço em dólares americanos-\\
R:(Atributo tipo numérico inteiro)\\
Coluna carat: peso do diamente-\\
R:(Atributo tipo numérico double com casas decemais)\\
Coluna cut:qualidade do corte (Fair, Good, Very Good, Premium, Ideal)-\\
R:(Atributo tipo Factor - categórico ou qualitativo)\\
Coluna color: cor do diamante, indo de J (pior) a D (melhor)-\\
R:(Atributo tipo Factor - categórico ou qualitativo)\\
Coluna clarity: medida de quão claro é o diamante-\\
R:(Atributo tipo Factor - categórico ou qualitativo)\\
Coluna depth: percentual de profundidade total-\\
R:(Atributo tipo numérico double com casas decemais)\\
Coluna table: largura do topo do diamante relativo ao ponto mais
largo-\\
R:(Atributo tipo numérico double com casas decemais)\\
Coluna x: comprimento em mm-\\
R:(Atributo tipo numérico double com casas decemais)\\
Coluna y: largura em mm-\\
R:(Atributo tipo numérico double com casas decemais)\\
Coluna z: profundidade em mm-\\
R:(Atributo tipo numérico double com casas decemais)

\hypertarget{a-sauxedda-da-funuxe7uxe3o-summary-estuxe1-de-acordo-com-a-descriuxe7uxe3o-mostrada-anteriormente}{%
\subsubsection{A saída da função summary() está de acordo com a
descrição mostrada
anteriormente?}\label{a-sauxedda-da-funuxe7uxe3o-summary-estuxe1-de-acordo-com-a-descriuxe7uxe3o-mostrada-anteriormente}}

As colunas que são do tipo numéricos (inteiros e/ou double com casas
decimais) e seus resultados estão de acordo com a função summary ,pois
apresntam tipos numéricos. Ex.:carat, price, depth, table, x,y,z.\\
Já as colunas que são do tipo caracteres (factor ordinais - são do tipo
categórico), não estão de acordo com a função summary.\\
Ex.:cut, color, clarity.

\hypertarget{resumo-do-banco-de-dados}{%
\subsection{Resumo do Banco de Dados}\label{resumo-do-banco-de-dados}}

Usamos esta biblioteca para visualizarmos de forma global a tabela e
suas colunas(variáveis).

\begin{Shaded}
\begin{Highlighting}[]
\CommentTok{#install.packages("skimr")}
\KeywordTok{library}\NormalTok{(skimr)}
\KeywordTok{skim}\NormalTok{(diamonds)}
\end{Highlighting}
\end{Shaded}

\begin{longtable}[]{@{}ll@{}}
\caption{Data summary}\tabularnewline
\toprule
\endhead
Name & diamonds\tabularnewline
Number of rows & 53940\tabularnewline
Number of columns & 10\tabularnewline
\_\_\_\_\_\_\_\_\_\_\_\_\_\_\_\_\_\_\_\_\_\_\_ &\tabularnewline
Column type frequency: &\tabularnewline
factor & 3\tabularnewline
numeric & 7\tabularnewline
\_\_\_\_\_\_\_\_\_\_\_\_\_\_\_\_\_\_\_\_\_\_\_\_ &\tabularnewline
Group variables & None\tabularnewline
\bottomrule
\end{longtable}

\textbf{Variable type: factor}

\begin{longtable}[]{@{}lrrlrl@{}}
\toprule
skim\_variable & n\_missing & complete\_rate & ordered & n\_unique &
top\_counts\tabularnewline
\midrule
\endhead
cut & 0 & 1 & TRUE & 5 & Ide: 21551, Pre: 13791, Ver: 12082, Goo:
4906\tabularnewline
color & 0 & 1 & TRUE & 7 & G: 11292, E: 9797, F: 9542, H:
8304\tabularnewline
clarity & 0 & 1 & TRUE & 8 & SI1: 13065, VS2: 12258, SI2: 9194, VS1:
8171\tabularnewline
\bottomrule
\end{longtable}

\textbf{Variable type: numeric}

\begin{longtable}[]{@{}lrrrrrrrrrl@{}}
\toprule
skim\_variable & n\_missing & complete\_rate & mean & sd & p0 & p25 &
p50 & p75 & p100 & hist\tabularnewline
\midrule
\endhead
carat & 0 & 1 & 0.80 & 0.47 & 0.2 & 0.40 & 0.70 & 1.04 & 5.01 &
▇▂▁▁▁\tabularnewline
depth & 0 & 1 & 61.75 & 1.43 & 43.0 & 61.00 & 61.80 & 62.50 & 79.00 &
▁▁▇▁▁\tabularnewline
table & 0 & 1 & 57.46 & 2.23 & 43.0 & 56.00 & 57.00 & 59.00 & 95.00 &
▁▇▁▁▁\tabularnewline
price & 0 & 1 & 3932.80 & 3989.44 & 326.0 & 950.00 & 2401.00 & 5324.25 &
18823.00 & ▇▂▁▁▁\tabularnewline
x & 0 & 1 & 5.73 & 1.12 & 0.0 & 4.71 & 5.70 & 6.54 & 10.74 &
▁▁▇▃▁\tabularnewline
y & 0 & 1 & 5.73 & 1.14 & 0.0 & 4.72 & 5.71 & 6.54 & 58.90 &
▇▁▁▁▁\tabularnewline
z & 0 & 1 & 3.54 & 0.71 & 0.0 & 2.91 & 3.53 & 4.04 & 31.80 &
▇▁▁▁▁\tabularnewline
\bottomrule
\end{longtable}

\begin{Shaded}
\begin{Highlighting}[]
\CommentTok{#O comando skim é interessante, porque além do sumário numérico, também temos um pequeno histograma para vermos a distribuição dos dados.  }
\CommentTok{# Vamos analisar as variáveis contínuas (numéricas) através dos histogramas. }
\end{Highlighting}
\end{Shaded}

\hypertarget{gruxe1fico-histograma}{%
\subsection{Gráfico Histograma}\label{gruxe1fico-histograma}}

\hypertarget{explorando-a-variuxe1vel-carat-atravuxe9s-do-histograma.}{%
\subsubsection{Explorando a variável carat, através do
histograma.}\label{explorando-a-variuxe1vel-carat-atravuxe9s-do-histograma.}}

\begin{Shaded}
\begin{Highlighting}[]
\KeywordTok{hist}\NormalTok{(diamonds}\OperatorTok{$}\NormalTok{carat)}
\end{Highlighting}
\end{Shaded}

\includegraphics{Explorando-a-Base-de-Dados-diamonds-do-pacote-ggplot2-Trilha3_files/figure-latex/Explorando a variável carat, através do histograma.-1.pdf}

\begin{Shaded}
\begin{Highlighting}[]
\KeywordTok{ggplot}\NormalTok{(diamonds, }\KeywordTok{aes}\NormalTok{(}\DataTypeTok{x =}\NormalTok{ carat)) }\OperatorTok{+}\StringTok{ }\KeywordTok{geom_histogram}\NormalTok{(}\DataTypeTok{bins=}\DecValTok{30}\NormalTok{, }\DataTypeTok{color =} \StringTok{"white"}\NormalTok{, }\DataTypeTok{fill =} \StringTok{"lightblue"}\NormalTok{)}
\end{Highlighting}
\end{Shaded}

\includegraphics{Explorando-a-Base-de-Dados-diamonds-do-pacote-ggplot2-Trilha3_files/figure-latex/Explorando a variável carat, através do histograma.-2.pdf}
\#\#\# Análise: Acima fizemos a comparação da distribuição dos pesos dos
diamantes conforme a faixa das quantidades, usando o histograma.

\hypertarget{explorando-a-variuxe1vel-depth-atravuxe9s-do-histograma.}{%
\subsubsection{Explorando a variável depth, através do
histograma.}\label{explorando-a-variuxe1vel-depth-atravuxe9s-do-histograma.}}

\begin{Shaded}
\begin{Highlighting}[]
\KeywordTok{hist}\NormalTok{(diamonds}\OperatorTok{$}\NormalTok{depth)}
\end{Highlighting}
\end{Shaded}

\includegraphics{Explorando-a-Base-de-Dados-diamonds-do-pacote-ggplot2-Trilha3_files/figure-latex/\#\#\# Explorando a variável depth, através do histograma.-1.pdf}

\begin{Shaded}
\begin{Highlighting}[]
\KeywordTok{ggplot}\NormalTok{(diamonds, }\KeywordTok{aes}\NormalTok{(}\DataTypeTok{x =}\NormalTok{ depth)) }\OperatorTok{+}\StringTok{ }\KeywordTok{geom_histogram}\NormalTok{(}\DataTypeTok{bins=}\DecValTok{30}\NormalTok{, }\DataTypeTok{color =} \StringTok{"white"}\NormalTok{, }\DataTypeTok{fill =} \StringTok{"green"}\NormalTok{)}
\end{Highlighting}
\end{Shaded}

\includegraphics{Explorando-a-Base-de-Dados-diamonds-do-pacote-ggplot2-Trilha3_files/figure-latex/\#\#\# Explorando a variável depth, através do histograma.-2.pdf}
\#\#\# Análise: Acima fizemos a comparação da distribuição das
profundidades (em porcentagem) dos diamantes conforme a faixa das
quantidades, usando o histograma.

\hypertarget{explorando-a-variuxe1vel-table-atravuxe9s-do-histograma.}{%
\subsubsection{Explorando a variável table, através do
histograma.}\label{explorando-a-variuxe1vel-table-atravuxe9s-do-histograma.}}

\begin{Shaded}
\begin{Highlighting}[]
\KeywordTok{hist}\NormalTok{(diamonds}\OperatorTok{$}\NormalTok{table)}
\end{Highlighting}
\end{Shaded}

\includegraphics{Explorando-a-Base-de-Dados-diamonds-do-pacote-ggplot2-Trilha3_files/figure-latex/Explorando a variável table, através do histograma.-1.pdf}

\begin{Shaded}
\begin{Highlighting}[]
\KeywordTok{ggplot}\NormalTok{(diamonds, }\KeywordTok{aes}\NormalTok{(}\DataTypeTok{x =}\NormalTok{ table)) }\OperatorTok{+}\StringTok{ }\KeywordTok{geom_histogram}\NormalTok{(}\DataTypeTok{bins =} \DecValTok{30}\NormalTok{, }\DataTypeTok{color =} \StringTok{"white"}\NormalTok{, }\DataTypeTok{fill =} \StringTok{"red"}\NormalTok{)}
\end{Highlighting}
\end{Shaded}

\includegraphics{Explorando-a-Base-de-Dados-diamonds-do-pacote-ggplot2-Trilha3_files/figure-latex/Explorando a variável table, através do histograma.-2.pdf}
\#\#\# Análise: Acima fizemos a comparação da distribuição das larguras
(mm) dos diamantes conforme a faixa das quantidades, usando o
histograma.

\hypertarget{explorando-a-variuxe1vel-price-atravuxe9s-do-histograma.}{%
\subsubsection{Explorando a variável price, através do
histograma.}\label{explorando-a-variuxe1vel-price-atravuxe9s-do-histograma.}}

\begin{Shaded}
\begin{Highlighting}[]
\KeywordTok{hist}\NormalTok{(diamonds}\OperatorTok{$}\NormalTok{price)}
\end{Highlighting}
\end{Shaded}

\includegraphics{Explorando-a-Base-de-Dados-diamonds-do-pacote-ggplot2-Trilha3_files/figure-latex/Explorando a variável price, através do histograma.-1.pdf}

\begin{Shaded}
\begin{Highlighting}[]
\KeywordTok{ggplot}\NormalTok{(diamonds, }\KeywordTok{aes}\NormalTok{(}\DataTypeTok{x =}\NormalTok{ price)) }\OperatorTok{+}\StringTok{ }\KeywordTok{geom_histogram}\NormalTok{(}\DataTypeTok{bins =} \DecValTok{30}\NormalTok{, }\DataTypeTok{color =} \StringTok{"white"}\NormalTok{, }\DataTypeTok{fill =} \StringTok{"orange"}\NormalTok{)}
\end{Highlighting}
\end{Shaded}

\includegraphics{Explorando-a-Base-de-Dados-diamonds-do-pacote-ggplot2-Trilha3_files/figure-latex/Explorando a variável price, através do histograma.-2.pdf}
\#\#\# Análise: Acima fizemos a comparação da distribuição dos preços
(dólares americanos) dos diamantes conforme a faixa das quantidades,
usando o histograma.

\#\#\#Explorando a variável x, através do histograma.

\begin{Shaded}
\begin{Highlighting}[]
\KeywordTok{hist}\NormalTok{(diamonds}\OperatorTok{$}\NormalTok{x)}
\end{Highlighting}
\end{Shaded}

\includegraphics{Explorando-a-Base-de-Dados-diamonds-do-pacote-ggplot2-Trilha3_files/figure-latex/Explorando a variável x, através do histograma.-1.pdf}

\begin{Shaded}
\begin{Highlighting}[]
\KeywordTok{ggplot}\NormalTok{(diamonds, }\KeywordTok{aes}\NormalTok{(}\DataTypeTok{x =}\NormalTok{ x)) }\OperatorTok{+}\StringTok{ }\KeywordTok{geom_histogram}\NormalTok{(}\DataTypeTok{bins=}\DecValTok{30}\NormalTok{, }\DataTypeTok{color =} \StringTok{"white"}\NormalTok{, }\DataTypeTok{fill =} \StringTok{"purple"}\NormalTok{)}
\end{Highlighting}
\end{Shaded}

\includegraphics{Explorando-a-Base-de-Dados-diamonds-do-pacote-ggplot2-Trilha3_files/figure-latex/Explorando a variável x, através do histograma.-2.pdf}
\#\#\# Análise: Acima fizemos a comparação da distribuição do
comprimento (mm) dos diamantes conforme a faixa das quantidades, usando
o histograma.

\#\#\#Explorando a variável y, através do histograma.

\begin{Shaded}
\begin{Highlighting}[]
\KeywordTok{hist}\NormalTok{(diamonds}\OperatorTok{$}\NormalTok{y)}
\end{Highlighting}
\end{Shaded}

\includegraphics{Explorando-a-Base-de-Dados-diamonds-do-pacote-ggplot2-Trilha3_files/figure-latex/Explorando a variável y, através do histograma.-1.pdf}

\begin{Shaded}
\begin{Highlighting}[]
\KeywordTok{ggplot}\NormalTok{(diamonds, }\KeywordTok{aes}\NormalTok{(}\DataTypeTok{x =}\NormalTok{ y)) }\OperatorTok{+}\StringTok{ }\KeywordTok{geom_histogram}\NormalTok{(}\DataTypeTok{bins=}\DecValTok{30}\NormalTok{, }\DataTypeTok{color =} \StringTok{"white"}\NormalTok{, }\DataTypeTok{fill =} \StringTok{"blue"}\NormalTok{)}
\end{Highlighting}
\end{Shaded}

\includegraphics{Explorando-a-Base-de-Dados-diamonds-do-pacote-ggplot2-Trilha3_files/figure-latex/Explorando a variável y, através do histograma.-2.pdf}
\#\#\# Análise: Acima fizemos a comparação da distribuição da largura
(mm) dos diamantes conforme a faixa das quantidades, usando o
histograma.

\#\#\#Explorando a variável z, através do histograma.

\begin{Shaded}
\begin{Highlighting}[]
\KeywordTok{hist}\NormalTok{(diamonds}\OperatorTok{$}\NormalTok{z)}
\end{Highlighting}
\end{Shaded}

\includegraphics{Explorando-a-Base-de-Dados-diamonds-do-pacote-ggplot2-Trilha3_files/figure-latex/Explorando a variável z, através do histograma.-1.pdf}

\begin{Shaded}
\begin{Highlighting}[]
\KeywordTok{ggplot}\NormalTok{(diamonds, }\KeywordTok{aes}\NormalTok{(}\DataTypeTok{x =}\NormalTok{ z)) }\OperatorTok{+}\StringTok{ }\KeywordTok{geom_histogram}\NormalTok{(}\DataTypeTok{bins =} \DecValTok{30}\NormalTok{, }\DataTypeTok{color =} \StringTok{"white"}\NormalTok{, }\DataTypeTok{fill =} \StringTok{"red"}\NormalTok{)}
\end{Highlighting}
\end{Shaded}

\includegraphics{Explorando-a-Base-de-Dados-diamonds-do-pacote-ggplot2-Trilha3_files/figure-latex/Explorando a variável z, através do histograma.-2.pdf}
\#\#\# Análise: Acima fizemos a comparação da distribuição da
profundidade (mm) dos diamantes conforme a faixa das quantidades, usando
o histograma.

\hypertarget{gruxe1fico-boxplot}{%
\subsection{Gráfico Boxplot}\label{gruxe1fico-boxplot}}

\hypertarget{explore-a-variuxe1vel-price-seguindo-o-modelo-de-explorauxe7uxe3o.}{%
\subsubsection{Explore a variável price, seguindo o modelo de
exploração.}\label{explore-a-variuxe1vel-price-seguindo-o-modelo-de-explorauxe7uxe3o.}}

Vamos analisar as variáveis de forma separada utilizando o gráfico
Boxplot.

\begin{Shaded}
\begin{Highlighting}[]
\CommentTok{#Boxplot para variáveis numéricas (devido a separação dos quartis)}
\KeywordTok{ggplot}\NormalTok{(}\DataTypeTok{data =}\NormalTok{ diamonds, }\DataTypeTok{mapping =} \KeywordTok{aes}\NormalTok{(}\DataTypeTok{x =}\NormalTok{ cut, }\DataTypeTok{y =}\NormalTok{ price, }\DataTypeTok{fill =}\NormalTok{cut)) }\OperatorTok{+}\StringTok{ }\KeywordTok{geom_boxplot}\NormalTok{(}\DataTypeTok{outlier.color =} \StringTok{"red"}\NormalTok{)}
\end{Highlighting}
\end{Shaded}

\includegraphics{Explorando-a-Base-de-Dados-diamonds-do-pacote-ggplot2-Trilha3_files/figure-latex/Explorando a variável price:-1.pdf}

\begin{Shaded}
\begin{Highlighting}[]
\CommentTok{#xlab("Qualidade do corte") + ylab("Preço(Dólares americanos)")}
\KeywordTok{ggplot}\NormalTok{(}\DataTypeTok{data =}\NormalTok{ diamonds, }\DataTypeTok{mapping =} \KeywordTok{aes}\NormalTok{(}\DataTypeTok{x =}\NormalTok{ color, }\DataTypeTok{y =}\NormalTok{ price, }\DataTypeTok{fill =}\NormalTok{color)) }\OperatorTok{+}\StringTok{ }\KeywordTok{geom_boxplot}\NormalTok{(}\DataTypeTok{outlier.color =} \StringTok{"red"}\NormalTok{)}
\end{Highlighting}
\end{Shaded}

\includegraphics{Explorando-a-Base-de-Dados-diamonds-do-pacote-ggplot2-Trilha3_files/figure-latex/Explorando a variável price:-2.pdf}

\begin{Shaded}
\begin{Highlighting}[]
\CommentTok{#xlab("Cor") + ylab("Preço(Dólares americanos)")}
\KeywordTok{ggplot}\NormalTok{(}\DataTypeTok{data =}\NormalTok{ diamonds, }\DataTypeTok{mapping =} \KeywordTok{aes}\NormalTok{(}\DataTypeTok{x =}\NormalTok{ clarity, }\DataTypeTok{y =}\NormalTok{ price, }\DataTypeTok{fill =}\NormalTok{clarity)) }\OperatorTok{+}\StringTok{ }\KeywordTok{geom_boxplot}\NormalTok{(}\DataTypeTok{outlier.color =} \StringTok{"red"}\NormalTok{)}
\end{Highlighting}
\end{Shaded}

\includegraphics{Explorando-a-Base-de-Dados-diamonds-do-pacote-ggplot2-Trilha3_files/figure-latex/Explorando a variável price:-3.pdf}

\begin{Shaded}
\begin{Highlighting}[]
\CommentTok{#xlab("Quão claro é o diamante") + ylab("Preço(Dólares americanos)")}
\end{Highlighting}
\end{Shaded}

\hypertarget{anuxe1lise}{%
\subsubsection{Análise:}\label{anuxe1lise}}

Acima fizemos a comparação dos preços dos diamantes conforme a qualidade
do corte, cor e claridade (transparencia do diamante). O tipo de corte
que mais influencia o preço é o corte premium. A qualidade de cor que
mais influencia o preço é a cor J (pior cor). A claridade
(transparência) que mais influencia o preço são as claridades VS1 e VS2.

\hypertarget{explore-a-variuxe1vel-carat-seguindo-o-modelo-de-explorauxe7uxe3o.}{%
\subsubsection{Explore a variável carat, seguindo o modelo de
exploração.}\label{explore-a-variuxe1vel-carat-seguindo-o-modelo-de-explorauxe7uxe3o.}}

\begin{Shaded}
\begin{Highlighting}[]
\KeywordTok{ggplot}\NormalTok{(}\DataTypeTok{data =}\NormalTok{ diamonds, }\DataTypeTok{mapping =} \KeywordTok{aes}\NormalTok{(}\DataTypeTok{x =}\NormalTok{ cut, }\DataTypeTok{y =}\NormalTok{ carat, }\DataTypeTok{fill =}\NormalTok{cut )) }\OperatorTok{+}\StringTok{ }\KeywordTok{geom_boxplot}\NormalTok{(}\DataTypeTok{outlier.color =} \StringTok{"red"}\NormalTok{)}
\end{Highlighting}
\end{Shaded}

\includegraphics{Explorando-a-Base-de-Dados-diamonds-do-pacote-ggplot2-Trilha3_files/figure-latex/Explorando a variável carat.-1.pdf}

\begin{Shaded}
\begin{Highlighting}[]
\KeywordTok{ggplot}\NormalTok{(}\DataTypeTok{data =}\NormalTok{ diamonds, }\DataTypeTok{mapping =} \KeywordTok{aes}\NormalTok{(}\DataTypeTok{x =}\NormalTok{ color, }\DataTypeTok{y =}\NormalTok{ carat, }\DataTypeTok{fill =}\NormalTok{color )) }\OperatorTok{+}\StringTok{ }\KeywordTok{geom_boxplot}\NormalTok{(}\DataTypeTok{outlier.color =} \StringTok{"red"}\NormalTok{)}
\end{Highlighting}
\end{Shaded}

\includegraphics{Explorando-a-Base-de-Dados-diamonds-do-pacote-ggplot2-Trilha3_files/figure-latex/Explorando a variável carat.-2.pdf}

\begin{Shaded}
\begin{Highlighting}[]
\KeywordTok{ggplot}\NormalTok{(}\DataTypeTok{data =}\NormalTok{ diamonds, }\DataTypeTok{mapping =} \KeywordTok{aes}\NormalTok{(}\DataTypeTok{x =}\NormalTok{ clarity, }\DataTypeTok{y =}\NormalTok{ carat, }\DataTypeTok{fill =}\NormalTok{clarity )) }\OperatorTok{+}\StringTok{ }\KeywordTok{geom_boxplot}\NormalTok{(}\DataTypeTok{outlier.color =} \StringTok{"red"}\NormalTok{)}
\end{Highlighting}
\end{Shaded}

\includegraphics{Explorando-a-Base-de-Dados-diamonds-do-pacote-ggplot2-Trilha3_files/figure-latex/Explorando a variável carat.-3.pdf}
\#\#\# Análise: Acima fizemos a comparação do peso dos diamantes
conforme a qualidade do corte, cor e claridade (transparencia do
diamante). O tipo de corte que mais influencia o peso é o corte premium.
A qualidade de cor que mais influencia o peso é a cor J (pior cor). A
claridade (transparência) que mais influencia o peso é a I1 (pior).

\hypertarget{explore-a-variuxe1vel-depth-seguindo-o-modelo-de-explorauxe7uxe3o.}{%
\subsubsection{Explore a variável depth, seguindo o modelo de
exploração.}\label{explore-a-variuxe1vel-depth-seguindo-o-modelo-de-explorauxe7uxe3o.}}

\begin{Shaded}
\begin{Highlighting}[]
\KeywordTok{ggplot}\NormalTok{(}\DataTypeTok{data =}\NormalTok{ diamonds, }\DataTypeTok{mapping =} \KeywordTok{aes}\NormalTok{(}\DataTypeTok{x =}\NormalTok{ cut, }\DataTypeTok{y =}\NormalTok{depth , }\DataTypeTok{fill =}\NormalTok{cut )) }\OperatorTok{+}\StringTok{ }\KeywordTok{geom_boxplot}\NormalTok{(}\DataTypeTok{outlier.color =} \StringTok{"red"}\NormalTok{)}
\end{Highlighting}
\end{Shaded}

\includegraphics{Explorando-a-Base-de-Dados-diamonds-do-pacote-ggplot2-Trilha3_files/figure-latex/Explorando a variável depth.-1.pdf}

\begin{Shaded}
\begin{Highlighting}[]
\KeywordTok{ggplot}\NormalTok{(}\DataTypeTok{data =}\NormalTok{ diamonds, }\DataTypeTok{mapping =} \KeywordTok{aes}\NormalTok{(}\DataTypeTok{x =}\NormalTok{ color, }\DataTypeTok{y =}\NormalTok{depth , }\DataTypeTok{fill =}\NormalTok{color )) }\OperatorTok{+}\StringTok{ }\KeywordTok{geom_boxplot}\NormalTok{(}\DataTypeTok{outlier.color =} \StringTok{"red"}\NormalTok{)}
\end{Highlighting}
\end{Shaded}

\includegraphics{Explorando-a-Base-de-Dados-diamonds-do-pacote-ggplot2-Trilha3_files/figure-latex/Explorando a variável depth.-2.pdf}

\begin{Shaded}
\begin{Highlighting}[]
\KeywordTok{ggplot}\NormalTok{(}\DataTypeTok{data =}\NormalTok{ diamonds, }\DataTypeTok{mapping =} \KeywordTok{aes}\NormalTok{(}\DataTypeTok{x =}\NormalTok{ clarity, }\DataTypeTok{y =}\NormalTok{depth , }\DataTypeTok{fill =}\NormalTok{clarity )) }\OperatorTok{+}\StringTok{ }\KeywordTok{geom_boxplot}\NormalTok{(}\DataTypeTok{outlier.color =} \StringTok{"red"}\NormalTok{)}
\end{Highlighting}
\end{Shaded}

\includegraphics{Explorando-a-Base-de-Dados-diamonds-do-pacote-ggplot2-Trilha3_files/figure-latex/Explorando a variável depth.-3.pdf}
\#\#\# Análise: Acima fizemos a comparação da porcentagem de
profundidade dos diamantes conforme a qualidade do corte, cor e
claridade (transparencia do diamante). O tipo de corte que mais
influencia o a profundidade é o Fair. A qualidade de cor que mais
influencia a profundidade E, G e J,pois apresentam valores outliers,
embora todas sejam muito parecidos em sua mediana. A claridade
(transparência) que mais influencia a profundidade é a I1, embora todas
sejam muito parecidos em sua mediana.

\hypertarget{explore-a-variuxe1vel-table-seguindo-o-modelo-de-explorauxe7uxe3o.}{%
\subsubsection{Explore a variável table, seguindo o modelo de
exploração.}\label{explore-a-variuxe1vel-table-seguindo-o-modelo-de-explorauxe7uxe3o.}}

\begin{Shaded}
\begin{Highlighting}[]
\KeywordTok{ggplot}\NormalTok{(}\DataTypeTok{data =}\NormalTok{ diamonds, }\DataTypeTok{mapping =} \KeywordTok{aes}\NormalTok{(}\DataTypeTok{x =}\NormalTok{ cut, }\DataTypeTok{y =}\NormalTok{table , }\DataTypeTok{fill =}\NormalTok{cut )) }\OperatorTok{+}\StringTok{ }\KeywordTok{geom_boxplot}\NormalTok{(}\DataTypeTok{outlier.color =} \StringTok{"red"}\NormalTok{)}
\end{Highlighting}
\end{Shaded}

\includegraphics{Explorando-a-Base-de-Dados-diamonds-do-pacote-ggplot2-Trilha3_files/figure-latex/Explorando a variável table-1.pdf}

\begin{Shaded}
\begin{Highlighting}[]
\KeywordTok{ggplot}\NormalTok{(}\DataTypeTok{data =}\NormalTok{ diamonds, }\DataTypeTok{mapping =} \KeywordTok{aes}\NormalTok{(}\DataTypeTok{x =}\NormalTok{ color, }\DataTypeTok{y =}\NormalTok{table , }\DataTypeTok{fill =}\NormalTok{color )) }\OperatorTok{+}\StringTok{ }\KeywordTok{geom_boxplot}\NormalTok{(}\DataTypeTok{outlier.color =} \StringTok{"red"}\NormalTok{)}
\end{Highlighting}
\end{Shaded}

\includegraphics{Explorando-a-Base-de-Dados-diamonds-do-pacote-ggplot2-Trilha3_files/figure-latex/Explorando a variável table-2.pdf}

\begin{Shaded}
\begin{Highlighting}[]
\KeywordTok{ggplot}\NormalTok{(}\DataTypeTok{data =}\NormalTok{ diamonds, }\DataTypeTok{mapping =} \KeywordTok{aes}\NormalTok{(}\DataTypeTok{x =}\NormalTok{ clarity, }\DataTypeTok{y =}\NormalTok{table , }\DataTypeTok{fill =}\NormalTok{clarity )) }\OperatorTok{+}\StringTok{ }\KeywordTok{geom_boxplot}\NormalTok{(}\DataTypeTok{outlier.color =} \StringTok{"red"}\NormalTok{)}
\end{Highlighting}
\end{Shaded}

\includegraphics{Explorando-a-Base-de-Dados-diamonds-do-pacote-ggplot2-Trilha3_files/figure-latex/Explorando a variável table-3.pdf}
\#\#\# Análise: Acima fizemos a comparação da largura do topo do
diamante relativo ao ponto mais largo dos diamantes conforme a qualidade
do corte, cor e claridade (transparencia do diamante). O tipo de corte
que mais influencia a largura do topo do diamante relativo ao ponto mais
largo é o corte Fair, embora as medianas sejam muito próximas. A
qualidade de cor que mais influencia a largura do topo do diamante
relativo ao ponto mais largo é a cor F, embora as medianas sejam muito
próximas. A claridade (transparência) que mais influencia largura do
topo do diamante relativo ao ponto mais largo é a transparência SI1,
embora as medianas sejam muito próximas.

\hypertarget{explore-a-variuxe1vel-z-seguindo-o-modelo-de-explorauxe7uxe3o.}{%
\subsubsection{Explore a variável z, seguindo o modelo de
exploração.}\label{explore-a-variuxe1vel-z-seguindo-o-modelo-de-explorauxe7uxe3o.}}

\begin{Shaded}
\begin{Highlighting}[]
\KeywordTok{ggplot}\NormalTok{(}\DataTypeTok{data =}\NormalTok{ diamonds, }\DataTypeTok{mapping =} \KeywordTok{aes}\NormalTok{(}\DataTypeTok{x =}\NormalTok{ cut, }\DataTypeTok{y =}\NormalTok{z , }\DataTypeTok{fill =}\NormalTok{cut )) }\OperatorTok{+}\StringTok{ }\KeywordTok{geom_boxplot}\NormalTok{(}\DataTypeTok{outlier.color =} \StringTok{"red"}\NormalTok{)}
\end{Highlighting}
\end{Shaded}

\includegraphics{Explorando-a-Base-de-Dados-diamonds-do-pacote-ggplot2-Trilha3_files/figure-latex/Explorando a variável z-1.pdf}

\begin{Shaded}
\begin{Highlighting}[]
\KeywordTok{ggplot}\NormalTok{(}\DataTypeTok{data =}\NormalTok{ diamonds, }\DataTypeTok{mapping =} \KeywordTok{aes}\NormalTok{(}\DataTypeTok{x =}\NormalTok{ color, }\DataTypeTok{y =}\NormalTok{z , }\DataTypeTok{fill =}\NormalTok{color )) }\OperatorTok{+}\StringTok{ }\KeywordTok{geom_boxplot}\NormalTok{(}\DataTypeTok{outlier.color =} \StringTok{"red"}\NormalTok{)}
\end{Highlighting}
\end{Shaded}

\includegraphics{Explorando-a-Base-de-Dados-diamonds-do-pacote-ggplot2-Trilha3_files/figure-latex/Explorando a variável z-2.pdf}

\begin{Shaded}
\begin{Highlighting}[]
\KeywordTok{ggplot}\NormalTok{(}\DataTypeTok{data =}\NormalTok{ diamonds, }\DataTypeTok{mapping =} \KeywordTok{aes}\NormalTok{(}\DataTypeTok{x =}\NormalTok{ clarity, }\DataTypeTok{y =}\NormalTok{z , }\DataTypeTok{fill =}\NormalTok{clarity )) }\OperatorTok{+}\StringTok{ }\KeywordTok{geom_boxplot}\NormalTok{(}\DataTypeTok{outlier.color =} \StringTok{"red"}\NormalTok{)}
\end{Highlighting}
\end{Shaded}

\includegraphics{Explorando-a-Base-de-Dados-diamonds-do-pacote-ggplot2-Trilha3_files/figure-latex/Explorando a variável z-3.pdf}
\#\#\# Análise: Acima fizemos a comparação da profundidade dos diamantes
conforme a qualidade do corte, cor e claridade (transparencia do
diamante). O tipo de corte que mais influencia a profundidade é o corte
premium, embora as medianas sejam muito próximas. A qualidade de cor que
mais influencia a profundidade é a cor J,embora as medianas sejam muito
próximas. A claridade (transparência) que mais influencia a profundidade
é a SI2, embora as medianas sejam muito próximas.

\hypertarget{explore-a-variuxe1vel-x-seguindo-o-modelo-de-explorauxe7uxe3o.}{%
\subsubsection{Explore a variável x, seguindo o modelo de
exploração.}\label{explore-a-variuxe1vel-x-seguindo-o-modelo-de-explorauxe7uxe3o.}}

\begin{Shaded}
\begin{Highlighting}[]
\KeywordTok{ggplot}\NormalTok{(}\DataTypeTok{data =}\NormalTok{ diamonds, }\DataTypeTok{mapping =} \KeywordTok{aes}\NormalTok{(}\DataTypeTok{x =}\NormalTok{ cut, }\DataTypeTok{y =}\NormalTok{x , }\DataTypeTok{fill =}\NormalTok{cut )) }\OperatorTok{+}\StringTok{ }\KeywordTok{geom_boxplot}\NormalTok{(}\DataTypeTok{outlier.color =} \StringTok{"red"}\NormalTok{)}
\end{Highlighting}
\end{Shaded}

\includegraphics{Explorando-a-Base-de-Dados-diamonds-do-pacote-ggplot2-Trilha3_files/figure-latex/Explorando a variável x-1.pdf}

\begin{Shaded}
\begin{Highlighting}[]
\KeywordTok{ggplot}\NormalTok{(}\DataTypeTok{data =}\NormalTok{ diamonds, }\DataTypeTok{mapping =} \KeywordTok{aes}\NormalTok{(}\DataTypeTok{x =}\NormalTok{ color, }\DataTypeTok{y =}\NormalTok{x , }\DataTypeTok{fill =}\NormalTok{color )) }\OperatorTok{+}\StringTok{ }\KeywordTok{geom_boxplot}\NormalTok{(}\DataTypeTok{outlier.color =} \StringTok{"red"}\NormalTok{)}
\end{Highlighting}
\end{Shaded}

\includegraphics{Explorando-a-Base-de-Dados-diamonds-do-pacote-ggplot2-Trilha3_files/figure-latex/Explorando a variável x-2.pdf}

\begin{Shaded}
\begin{Highlighting}[]
\KeywordTok{ggplot}\NormalTok{(}\DataTypeTok{data =}\NormalTok{ diamonds, }\DataTypeTok{mapping =} \KeywordTok{aes}\NormalTok{(}\DataTypeTok{x =}\NormalTok{ clarity, }\DataTypeTok{y =}\NormalTok{x , }\DataTypeTok{fill =}\NormalTok{clarity )) }\OperatorTok{+}\StringTok{ }\KeywordTok{geom_boxplot}\NormalTok{(}\DataTypeTok{outlier.color =} \StringTok{"red"}\NormalTok{)}
\end{Highlighting}
\end{Shaded}

\includegraphics{Explorando-a-Base-de-Dados-diamonds-do-pacote-ggplot2-Trilha3_files/figure-latex/Explorando a variável x-3.pdf}
\#\#\# Análise: Acima fizemos a comparação do comprimento dos diamantes
conforme a qualidade do corte, cor e claridade (transparencia do
diamante). O tipo de corte que mais influencia o comprimento dos
diamantes é o premium, embora as medianas sejam muito próximas. A
qualidade de cor que mais influencia o comprimento dos diamantes é a cor
J (pior cor),embora as medianas sejam muito próximas. A claridade
(transparência) que mais influencia o comprimento dos diamantes é a I1,
embora todas sejam muito parecidos em sua mediana.

\hypertarget{explore-a-variuxe1vel-y-seguindo-o-modelo-de-explorauxe7uxe3o.}{%
\subsubsection{Explore a variável y, seguindo o modelo de
exploração.}\label{explore-a-variuxe1vel-y-seguindo-o-modelo-de-explorauxe7uxe3o.}}

\begin{Shaded}
\begin{Highlighting}[]
\KeywordTok{ggplot}\NormalTok{(}\DataTypeTok{data =}\NormalTok{ diamonds, }\DataTypeTok{mapping =} \KeywordTok{aes}\NormalTok{(}\DataTypeTok{x =}\NormalTok{ cut, }\DataTypeTok{y =}\NormalTok{y , }\DataTypeTok{fill =}\NormalTok{cut )) }\OperatorTok{+}\StringTok{ }\KeywordTok{geom_boxplot}\NormalTok{(}\DataTypeTok{outlier.color =} \StringTok{"red"}\NormalTok{)}
\end{Highlighting}
\end{Shaded}

\includegraphics{Explorando-a-Base-de-Dados-diamonds-do-pacote-ggplot2-Trilha3_files/figure-latex/Explorando a variável y-1.pdf}

\begin{Shaded}
\begin{Highlighting}[]
\KeywordTok{ggplot}\NormalTok{(}\DataTypeTok{data =}\NormalTok{ diamonds, }\DataTypeTok{mapping =} \KeywordTok{aes}\NormalTok{(}\DataTypeTok{x =}\NormalTok{ color, }\DataTypeTok{y =}\NormalTok{y , }\DataTypeTok{fill =}\NormalTok{color )) }\OperatorTok{+}\StringTok{ }\KeywordTok{geom_boxplot}\NormalTok{(}\DataTypeTok{outlier.color =} \StringTok{"red"}\NormalTok{)}
\end{Highlighting}
\end{Shaded}

\includegraphics{Explorando-a-Base-de-Dados-diamonds-do-pacote-ggplot2-Trilha3_files/figure-latex/Explorando a variável y-2.pdf}

\begin{Shaded}
\begin{Highlighting}[]
\KeywordTok{ggplot}\NormalTok{(}\DataTypeTok{data =}\NormalTok{ diamonds, }\DataTypeTok{mapping =} \KeywordTok{aes}\NormalTok{(}\DataTypeTok{x =}\NormalTok{ clarity, }\DataTypeTok{y =}\NormalTok{y , }\DataTypeTok{fill =}\NormalTok{clarity )) }\OperatorTok{+}\StringTok{ }\KeywordTok{geom_boxplot}\NormalTok{(}\DataTypeTok{outlier.color =} \StringTok{"red"}\NormalTok{)}
\end{Highlighting}
\end{Shaded}

\includegraphics{Explorando-a-Base-de-Dados-diamonds-do-pacote-ggplot2-Trilha3_files/figure-latex/Explorando a variável y-3.pdf}
\#\#\# Análise: Acima fizemos a comparação da largura dos diamantes
conforme a qualidade do corte, cor e claridade (transparencia do
diamante). O tipo de corte que mais influencia a largura dos diamantes é
o premium, embora as medianas sejam muito próximas. A qualidade de cor
que mais influencia a largura dos diamantes é a cor J (pior cor),embora
as medianas sejam muito próximas. A claridade (transparência) que mais
influencia a largura dos diamantes é a I1, embora todas sejam muito
parecidos em sua mediana.

\hypertarget{gruxe1fico-de-barras}{%
\subsection{Gráfico de Barras}\label{gruxe1fico-de-barras}}

\begin{Shaded}
\begin{Highlighting}[]
\KeywordTok{ggplot}\NormalTok{(diamonds, }\KeywordTok{aes}\NormalTok{(}\DataTypeTok{y =}\NormalTok{ price, }\DataTypeTok{x =}\NormalTok{ cut)) }\OperatorTok{+}\StringTok{ }\KeywordTok{geom_bar}\NormalTok{(}\DataTypeTok{stat =} \StringTok{"identity"}\NormalTok{, }\DataTypeTok{fill =} \StringTok{"tomato"}\NormalTok{)}
\end{Highlighting}
\end{Shaded}

\includegraphics{Explorando-a-Base-de-Dados-diamonds-do-pacote-ggplot2-Trilha3_files/figure-latex/Explorando a variável preço-1.pdf}

\begin{Shaded}
\begin{Highlighting}[]
\KeywordTok{ggplot}\NormalTok{(diamonds, }\KeywordTok{aes}\NormalTok{(}\DataTypeTok{y =}\NormalTok{ price, }\DataTypeTok{x =}\NormalTok{ color)) }\OperatorTok{+}\StringTok{ }\KeywordTok{geom_bar}\NormalTok{(}\DataTypeTok{stat =} \StringTok{"identity"}\NormalTok{, }\DataTypeTok{fill =} \StringTok{"blue"}\NormalTok{)}
\end{Highlighting}
\end{Shaded}

\includegraphics{Explorando-a-Base-de-Dados-diamonds-do-pacote-ggplot2-Trilha3_files/figure-latex/Explorando a variável preço-2.pdf}

\begin{Shaded}
\begin{Highlighting}[]
\KeywordTok{ggplot}\NormalTok{(diamonds, }\KeywordTok{aes}\NormalTok{(}\DataTypeTok{y =}\NormalTok{ price, }\DataTypeTok{x =}\NormalTok{ clarity)) }\OperatorTok{+}\StringTok{ }\KeywordTok{geom_bar}\NormalTok{(}\DataTypeTok{stat =} \StringTok{"identity"}\NormalTok{, }\DataTypeTok{fill =} \StringTok{"yellow"}\NormalTok{)}
\end{Highlighting}
\end{Shaded}

\includegraphics{Explorando-a-Base-de-Dados-diamonds-do-pacote-ggplot2-Trilha3_files/figure-latex/Explorando a variável preço-3.pdf}
\#\#\# Análise: Acima fizemos a comparação do preço dos diamantes
conforme a qualidade do corte, cor e claridade (transparencia do
diamante). O tipo de corte que mais influencia o preço dos diamantes é o
Ideal. A qualidade de cor que mais influencia o preço dos diamantes é a
cor G. A claridade (transparência) que mais influencia o preço dos
diamantes é o SI1.

\hypertarget{scatterplot-gruxe1fico-de-dispersao}{%
\subsection{Scatterplot (Gráfico de
Dispersao)}\label{scatterplot-gruxe1fico-de-dispersao}}

\begin{Shaded}
\begin{Highlighting}[]
\KeywordTok{ggplot}\NormalTok{(diamonds, }\KeywordTok{aes}\NormalTok{(}\DataTypeTok{x=}\NormalTok{price, }\DataTypeTok{y=}\NormalTok{carat)) }\OperatorTok{+}\StringTok{ }\KeywordTok{geom_point}\NormalTok{(}\DataTypeTok{color=}\StringTok{"yellow"}\NormalTok{)}
\end{Highlighting}
\end{Shaded}

\includegraphics{Explorando-a-Base-de-Dados-diamonds-do-pacote-ggplot2-Trilha3_files/figure-latex/Explorando a variável preço com o peso x largura x profundidade-1.pdf}

\begin{Shaded}
\begin{Highlighting}[]
\KeywordTok{ggplot}\NormalTok{(diamonds, }\KeywordTok{aes}\NormalTok{(}\DataTypeTok{x=}\NormalTok{price, }\DataTypeTok{y=}\NormalTok{y)) }\OperatorTok{+}\StringTok{ }\KeywordTok{geom_point}\NormalTok{(}\DataTypeTok{color=}\StringTok{"blue"}\NormalTok{)}
\end{Highlighting}
\end{Shaded}

\includegraphics{Explorando-a-Base-de-Dados-diamonds-do-pacote-ggplot2-Trilha3_files/figure-latex/Explorando a variável preço com o peso x largura x profundidade-2.pdf}

\begin{Shaded}
\begin{Highlighting}[]
\KeywordTok{ggplot}\NormalTok{(diamonds, }\KeywordTok{aes}\NormalTok{(}\DataTypeTok{x=}\NormalTok{price, }\DataTypeTok{y=}\NormalTok{z)) }\OperatorTok{+}\StringTok{ }\KeywordTok{geom_point}\NormalTok{(}\DataTypeTok{color=}\StringTok{"green"}\NormalTok{)}
\end{Highlighting}
\end{Shaded}

\includegraphics{Explorando-a-Base-de-Dados-diamonds-do-pacote-ggplot2-Trilha3_files/figure-latex/Explorando a variável preço com o peso x largura x profundidade-3.pdf}

\begin{Shaded}
\begin{Highlighting}[]
\KeywordTok{ggplot}\NormalTok{(diamonds, }\KeywordTok{aes}\NormalTok{(}\DataTypeTok{x=}\NormalTok{price, }\DataTypeTok{y=}\NormalTok{x)) }\OperatorTok{+}\StringTok{ }\KeywordTok{geom_point}\NormalTok{(}\DataTypeTok{color=}\StringTok{"purple"}\NormalTok{)}
\end{Highlighting}
\end{Shaded}

\includegraphics{Explorando-a-Base-de-Dados-diamonds-do-pacote-ggplot2-Trilha3_files/figure-latex/Explorando a variável preço com o peso x largura x profundidade-4.pdf}

\begin{Shaded}
\begin{Highlighting}[]
\KeywordTok{ggplot}\NormalTok{(diamonds, }\KeywordTok{aes}\NormalTok{(}\DataTypeTok{x=}\NormalTok{price, }\DataTypeTok{y=}\NormalTok{depth)) }\OperatorTok{+}\StringTok{ }\KeywordTok{geom_point}\NormalTok{(}\DataTypeTok{color=}\StringTok{"orange"}\NormalTok{)}
\end{Highlighting}
\end{Shaded}

\includegraphics{Explorando-a-Base-de-Dados-diamonds-do-pacote-ggplot2-Trilha3_files/figure-latex/Explorando a variável preço com o peso x largura x profundidade-5.pdf}

\begin{Shaded}
\begin{Highlighting}[]
\CommentTok{#preço x peso x corte}
\KeywordTok{ggplot}\NormalTok{ (diamonds, }\KeywordTok{aes}\NormalTok{ (}\DataTypeTok{x =}\NormalTok{ price, }\DataTypeTok{y =}\NormalTok{ carat)) }\OperatorTok{+}\StringTok{ }\KeywordTok{geom_point}\NormalTok{ (}\KeywordTok{aes}\NormalTok{ (}\DataTypeTok{color =} \KeywordTok{factor}\NormalTok{ (cut)))}
\end{Highlighting}
\end{Shaded}

\includegraphics{Explorando-a-Base-de-Dados-diamonds-do-pacote-ggplot2-Trilha3_files/figure-latex/unnamed-chunk-8-1.pdf}

\begin{Shaded}
\begin{Highlighting}[]
\CommentTok{#preço x comprimento x cor}
\KeywordTok{ggplot}\NormalTok{ (diamonds, }\KeywordTok{aes}\NormalTok{ (}\DataTypeTok{x =}\NormalTok{ price, }\DataTypeTok{y =}\NormalTok{ x)) }\OperatorTok{+}\StringTok{ }\KeywordTok{geom_point}\NormalTok{ (}\KeywordTok{aes}\NormalTok{ (}\DataTypeTok{color =} \KeywordTok{factor}\NormalTok{ (color)))}
\end{Highlighting}
\end{Shaded}

\includegraphics{Explorando-a-Base-de-Dados-diamonds-do-pacote-ggplot2-Trilha3_files/figure-latex/unnamed-chunk-8-2.pdf}

\begin{Shaded}
\begin{Highlighting}[]
\CommentTok{#preço x largura x "transparencia ou claridade"}
\KeywordTok{ggplot}\NormalTok{ (diamonds, }\KeywordTok{aes}\NormalTok{ (}\DataTypeTok{x =}\NormalTok{ price, }\DataTypeTok{y =}\NormalTok{ table)) }\OperatorTok{+}\StringTok{ }\KeywordTok{geom_point}\NormalTok{ (}\KeywordTok{aes}\NormalTok{ (}\DataTypeTok{color =} \KeywordTok{factor}\NormalTok{ (clarity)))}
\end{Highlighting}
\end{Shaded}

\includegraphics{Explorando-a-Base-de-Dados-diamonds-do-pacote-ggplot2-Trilha3_files/figure-latex/unnamed-chunk-8-3.pdf}

\end{document}
